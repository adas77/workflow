\documentclass[a4paper, 12pt]{article}
\usepackage[top=3cm, bottom=3cm, left = 2cm, right = 2cm]{geometry} 
\usepackage{float}
\geometry{a4paper} 
\usepackage[utf8]{inputenc}
\usepackage[T1]{fontenc}
\usepackage{textcomp}
\usepackage{graphicx} 
\usepackage{amsmath,amssymb}  
\usepackage{bm}  
\usepackage[pdftex,bookmarks,colorlinks,breaklinks]{hyperref}  
\usepackage{memhfixc} 
\usepackage{pdfsync}  
\usepackage{fancyhdr}

\pagestyle{fancy}

\title{Internetowy system zarządzania zadaniami pracowniczymi}
\author{Adam Pilewski}

\begin{document}
\maketitle
\tableofcontents
\section{Wstęp}
\subsection{Link do reppzytorium}
\href{https://github.com/adas77/workflow}{Link do kodu projektu na github}
\subsection{Technologie}
\begin{enumerate}
    \item Typescript
    \item Framework - \href{https://nextjs.org/}{Nextjs} (React)
    \item Baza danych \begin{enumerate}
        \item PostgreSQL
        \item ORM - \href{https://www.prisma.io/}{Prisma}
    \end{enumerate}
    \item API \begin{enumerate}
        \item głównie \href{https://trpc.io/}{tRPC}
        \item REST dla uploadu plików
    \end{enumerate}
    \item \href{https://docs.docker.com/compose/}{Docker Compose}
\end{enumerate}
\subsection{Krótki opis}
Do tej pory skupiłem się bardziej na funkcjonalności niż ładnym interfejsie - stworzyłem tylko podstawowe komponenty - które pozwolą na interakcję z systemem

\pagebreak

\section{Integracja z Google}
Głównie za pośrednictwem biblioteki \href{https://www.npmjs.com/package/googleapis}{googleapis} oraz \href{https://www.npmjs.com/package/nodemailer}{nodemailer} do wysyłania maili.
\subsection{Autoryzacja}
\begin{figure}[H]
	\centering
	\includegraphics*[scale=0.2]{img/google-auth.png}
	\caption{Niezalogowany użytkownik}
\end{figure}
\begin{figure}[H]
	\centering
    \includegraphics*[scale=0.4]{img/google-auth-permissions.png}
	\caption{Użytkownik wyraza zgode na dostęp aplikacji do jego kalendarza google i danych do autoryzacji}
\end{figure}

\subsection{Wysyłanie emaili}
\begin{figure}[H]
	\centering
    \includegraphics*[scale=0.4]{img/send-email.png}
	\caption{Użytkownik wysyła maila do osoby-osób}
\end{figure}
\begin{figure}[H]
	\centering
    \includegraphics*[scale=0.3]{img/send-email-2.png}
	\caption{Rezultat}
\end{figure}

\subsection{Kalendarz i Google Meets}
\begin{figure}[H]
	\centering
    \includegraphics*[scale=0.8]{img/create-task.png}
	\caption{Użytkownik tworzy zadania dla konkretnych użytkwników - może je także zapisać w kalendarzu Google'a}
\end{figure}
\begin{figure}[H]
	\centering
    \includegraphics*[scale=0.6]{img/create-task-2.png}
	\caption{Rezultat - każda z osób przydzielona do zadania - widzi je w kalendarzu i może dołączyć do spotkania}
\end{figure}
\begin{figure}[H]
	\centering
    \includegraphics*[scale=0.3]{img/google-meet.png}
	\caption{Dołączanie do spotkania}
\end{figure}
\begin{figure}[H]
	\centering
    \includegraphics*[scale=0.2]{img/google-meet-2.png}
    \caption{2 użytkowników dołączyło do rozmowy}
\end{figure}


\subsection{Podsumowanie}
\begin{figure}[H]
	\centering
    \includegraphics*[scale=0.4]{img/google-msgs.png}
	\caption{Przykład powiadomień na skrzynce Google'a - odwołanie zadania, otrzymanie maila, informacja o przekazaniu zezwoleń do Googla dla aplikacji}
\end{figure}

\section{Kanban Table}
Przykładowy widok zadań - zadanie jest w danym statusie - przesunięcie powoduje zmianę statusu zadania, po kliknięciu na zadanie widać szczegółowe informacje - usunąć zadanie może tylko osoba która je stworzyła
\begin{figure}[H]
	\centering
    \includegraphics*[scale=0.4]{img/kanban.png}
	\caption{Widok Kanban Board}
\end{figure}
\begin{figure}[H]
	\centering
    \includegraphics*[scale=0.2]{img/task-send-files.png}
	\caption{Szczegółowy widok zadania i upload plików}
\end{figure}
\begin{figure}[H]
	\centering
    \includegraphics*[scale=0.7]{img/task-send-files-2.png}
	\caption{Upload plików - pliki przechowywane na serwerze}
\end{figure}
\begin{figure}[H]
	\centering
    \includegraphics*[scale=0.3]{img/delete-task.png}
	\caption{Przykład usuwania zadań - jeśli osoba ma uprawnienia, może usunąć, w przeciwnym wypadku odmowa}
\end{figure}
\section{TODO na najbliższe tygodnie}
\begin{enumerate}
    \item Interfejs użytkownika \begin{enumerate}
        \item Dodanie komponentów
        \item Dopracowanie starych komponentów
        \item Dopracowanie widoków
    \end{enumerate}
        \item Dodanie przepływu zadań
        \item Czat między użytokwnikami
        \item Widok konkretnego zadania - historia plików
        \item Widok kalendarza
\end{enumerate}

\bibliographystyle{abbrv}

\end{document}